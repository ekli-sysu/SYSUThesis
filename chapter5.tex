% !Mode:: "TeX:UTF-8"

\chapter{总结及展望}
%结束语:画龙点睛
%归纳全文的创新点
%指出下一步需要开展的工作
%约3页左右,3000字左右

%总结
本文对极端质量比旋近引力波信号探测问题进行调研,采用卷积神经网络算法进行了初步的信号探测工作,主要分为构建模拟信号、模拟噪声、训练卷积神经网络算法和分析结果。
关于极端质量比旋近引力波模拟信号的构建,采用AK波形模型和AAK波形进行波形模拟,其中也采用波源天文学模型等产生波源参数等工作。而探测器仪器噪声采用天琴的噪声曲线模拟得到的高斯噪声。
对于信号探测算法,采用了卷积神经网络算法。训练一个卷积神经网络模型,需要构建三种数据集,包括训练数据、验证数据和测试数据,每种数据集含有一半模拟信号和一半模拟噪声。由于EMRI波形产生时间耗时相对较长和训练数据数目要求比较高,故而也采用了数据增强的方法,对训练数据进行了数据增广。
经过训练得到的卷积神经网络模型能够有效探测信噪比大于50的EMRI信号,且采用不同源参数分布和波形模型模拟得到模拟信号也能实现一致的探测结果。
% 信噪比结果
EMRI信号信噪比越高,模型得到的真值率越高。
% 内禀参数结果
给定参考波源(中心大质量黑洞$10^6 \rm M_{\odot}$, 自旋0.8, 红移0.1),分别分析了该模型对中心大质量黑洞质量、中心大质量黑洞自旋和红移等源参数不同的模拟信号的灵敏度。
% 光度距离结果
在固定误报率为1\%的情况下,中心大质量黑洞$10^6\rm M_{\odot}$的波源模拟的EMRI信号真值率最高;且红移为0.1时,真值率为90\%。同样的误报率条件下,带有自旋中心大质量黑洞构成的EMRI波源得到的模拟信号真值率也比较高,参考红移为0.1,得到的真值率为70\%。最后分析了参考波源在不同红移条件下,该类模拟信号得到的真值率是逐步减小,固定误报率为1\%, 该模型最远能探测红移为0.25的信号($\emph{i.e.}$ 真值率大于50\%)。

%创新性


%未来工作
针对当前工作,尽管采用深度学习算法能够提升探测速率,但是在效率上还有巨大的提升空间。具体来说,
采用卷积神经网络算法等机器学习算法来进行EMRI信号探测,这是基于数据驱动的方法,优点是避免了人工分析特征统计量的工作,实现了自动提取信号特征,由此达到端到端的处理过程。同时,也因为是数据驱动整个模型实现过程,故而需要大量的数据,且模型复杂度较高,训练相对困难。针对上述问题,未来工作希望从两方面来提升极端质量比旋近引力波探测效率。

针对第一点问题,加上考虑EMRI产生波形速率相对较慢,未来希望利用GAN算法对产生大量细微对抗样本,增加信号样本空间,这将对未来探测流水线的设计提供另一种尝试思路。

针对第二点问题,考虑分层搜索算法(Hierarchical search method),或者模型集成(Model Boosting)方法,降低对单一算法的准确率要求。