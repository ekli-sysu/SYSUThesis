% !Mode:: "TeX:UTF-8"
%% This is abstract

\begin{cabstract}
广义相对论的检验以及对其他修改引力理论的研究一直是近百年来物理学研究的重点。近年来,自LIGO第一次探测到双黑洞并合产生的引力波起,人们对这方面研究的兴趣日益增长。这是因为强引力场中的天体并合,例如黑洞和中子星,所产生的引力波中包含了丰富的波源信息,使得人们可以通过对这些信息的分析来更深一步地理解引力理论并进行实验验证。这些信息中,双星系统绕转时产生的旋近信号及最终并合时产生的铃宕信号都受到了很高的关注,我们将主要研究铃宕信号在检验广义相对论方面的作用。

我们具体研究了利用双黑洞并合产生的铃宕信号来限制特定修改引力理论和检验黑洞无毛定理。一方面,由于修改引力理论会导致相同质量与自旋参数下的黑洞准正则模出现偏离,我们计算了特定的一种理论(STVG,即标量-张量-矢量引力)中旋转黑洞的准正则模。利用其生成铃宕波形之后,我们结合具体的空间引力波探测器天琴与LISA,计算了在不同天文学模型下它们对修改理论中额外参数的限制能力。另一方面,我们对准正则模的一般偏离参数进行参数估计,并计算了它们对广义相对论中黑洞无毛定理的具体限制能力。同样的,在这个工作中我们也考虑了不同的探测器与天文学模型。

本文也将具体介绍各类修改引力理论、准正则模的计算方法以及参数估计的方法,并由此计算具体的探测器限制能力。我们进一步分析了这些结果,并希望这些结果能够对将来的引力波探测器的科学任务计划给予一定的帮助。

\ckeywords{引力波,黑洞,极端质量比旋近,卷积神经网络,深度学习}
\end{cabstract}

\begin{eabstract}
Testing general relativity and constraining modified theories of gravity are always the research emphases over the past century. Recently, people's interest about them is growing since LIGO detected the first event of gravitational waves from the merger of binary black holes. This is because the rich information encoded in the gravitational waves, which are generated by the celestial bodies in the strong gravitational fields like black holes and neutron stars, will help people to understand more about gravitational theories and test them experimentally. In this information, the inspiral signals from rotating binary stars and the ringdown signals from the last stage of the merger receive particular attention. We will focus on the ringdown signals and study how to use them to do the test.

Specifically, we study how to use the ringdown signals from the merger of binary black holes to constrain modified theory of gravity and test no-hair theorem of black hole. On the one hand, since the modification of gravitational theory will cause the deviation of quasi-normal modes for the black holes with same mass and spin, we calculate the quasi-normal modes of rotating black hole in STVG (Scalar-Tensor-Vector Gravity) theory. Then we generate the ringdown waveform by these modes, together with the space-borne gravitational wave detector LISA and TianQin, we study the ability of them to constrain the extra parameter in STVG under different astronomical models. On the other hand, with same detectors and astronomical models, we study their parameter estimation abilities of the theory-independent parameters of quasi-normal modes. These abilities correspond to the abilities of the detectors to constrain no-hair theorem of black holes in general relativity.

Besides, we will also introduce several typical modified theories of gravity, and methods to calculate the quasi-normal modes and parameter estimation accuracy, which are important for evaluating the precision of test. We hope these results will be helpful to the future scientific plans of the gravitational wave detectors.

\ekeywords{gravitational waves, quasi-normal modes, black hole, modified theory of gravity}
\end{eabstract}
