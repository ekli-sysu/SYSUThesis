% !Mode:: "TeX:UTF-8"

\markboth{致\quad 谢}{致\quad 谢}
\addcontentsline{toc}{chapter}{致\quad 谢} % 添加到目录中
\chapter*{致\quad 谢}
几年的研究生生活过去很快,就好像我刚刚下水游泳,还想畅游,就得上岸了。这几年困难苦恼过,也有很开心很开心的时刻。值此经历过的人事一并感谢。我会一直心怀感激,铭记在心。

在天琴中心的理论组,每个老师都很好,我从他们那里学到了很多东西。梅健伟老师是一个严格和善的人。曾有人跟我说梅老师是一个有理想信念的人,他衷心希望天琴能够成功,并且也坚定地践行着。在组内学习的这段时间里,我也能真切感受到这份信念的力量。他教了我往更深层次地思考和把握关键问题。
%胡老师
胡老师是个``半矛盾体"。一方面希望学生的我们不要太纠结于某个地方(比如推导某些阈值),加快工作进展。另一方面,又希望我们能正确把握原理,物理图景应该非常明晰。显然在老师那,他很清楚中间的平衡度在哪里,而我们却会在某个计算过程中一直胶着,然后就又上演了讲方法论的画面。不过,胡老师一直把我们学生放在平等的位置上,偶尔总是能够听到别人称呼老师叫”老胡“
。这就大概能可见一角了。于我个人而言,胡老师一直给我信心和信任。这一点超越了很多。
%张建东老师
张建东老师十分耐心、脾气平和。在我开始进组之后,我有许多概念与原理、计算等不懂、不确定的问题,我都能在跟张老师的讨论中得到答案。尤其在写文档时,因为跟老师的沟通,总能获得写作的新思路。有首歌叫"Star ",这很能形容我当时的心情,送给张建东老师。
%张雪峰老师
%石
张雪峰老师和石常富师兄脾气很好。在我迷茫的时候,他们也都开解过我。总会在某些时刻很迷茫,他们用他们的故事和经历也鼓舞着我继续往前走。谢谢他们。
%李恩坤
李恩坤老师认识不久,但李老师也在我毕业这手忙脚乱的时候,帮我解决了很多我不熟悉的问题,一一上手解决。日常上,只要我对课题有些思考都可以找机会跟李老师多讨论,他也会听我在"天马行空”。李老师也是劳模,会因为我的一个问题,晚上11点还给我解决方案。
说起劳模,我们组的老师真的都是劳动模范。按照佛学《现观庄严论》中所说:“发心为利他,求正等菩提。”

%范、罗、景、高、梁、朱、刘帅、
感谢范会敏、罗成健、景艺德、高洁、朱良贵、刘帅、梁正程等师兄师姐,有你们的帮助,我才能顺顺利利度过诸多麻烦和烦恼。谢谢2018级研究班的各位同学的配合和支持,让我担任了3年团支书工作能顺利交差,还能与你们一起拿到"中山大学五四红旗团支部", 特别感谢陈洋、杨书望、纪沐婕、谢宁、毛秋丽、严育宜、王观芳等同学。谢谢张开翼和周思晗,我偶有疏忽和错误,他们提醒我改正。

%感谢家人和朋友
谢谢一直陪伴在我身边的家人,尽管我可能一周都没有打一个电话给你们。谢谢宽容的妈妈、可爱的妹妹、操心的弟弟和令人头大的爸爸,至少你们身体健康,我们一直在一起努力,让一家人过得更好。谢谢男朋友黄国铨,恭喜他也顺利考上研究生。


研究生三年里,尽管我很讨厌那么多行政的条条框框,但是天琴中心诸多老师和同学都给了我很多善意的关怀和鼓励,这依旧是我人生中难忘的一段经历。衷心祝愿各位越来越好,天琴越来越好。